\chapter{Método dos elementos espectrais}
\label{cap:II}

\section{Formulação matemática}
 Para uma nova formulação do problema, tomemos novas notações, começando pela equação de Helmholtz de primeira ordem:
\begin{equation}
 L(u) = \frac{\partial^2 u}{\partial x^2} - \lambda u + f = 0
\end{equation}
 Onde u respeita as condições de contorno:\\
\begin{equation}
 u(0) = g_D \ , \ \frac{\partial u}{\partial x}(l) = g_N
\end{equation}
 Assim, temos que determinar a solução no intervalo $ 0 < x < l$, que será o domínio $\Omega$.
\begin{equation}
 \Omega = \{\ x\ |\ 0\ <\ x <\ l\} = ]0,1]
\end{equation}
 Multiplicando $L(u)$ por uma função peso $v(x)$ qualquer e de integrando no domínio $\Omega$, temos:
 \begin{equation}
 \int^{l}_0 v \frac{\partial^2 u}{\partial x^2} \partial x - \int^{l}_0 \lambda v u \partial x + \int^{l}_0 v f \partial x = 0
 \end{equation}
 Sendo $u(x)$ e $v(x)$ funções suaves, podemos integrar o primeiro termo por partes obtendo:
 \begin{equation}\label{eq:nova_eqdif}
 \int_\Omega \frac{\partial v}{\partial x} \frac{\partial u}{\partial x} \partial x\ + \int_\Omega \lambda v u \partial x\ =\ \int_\Omega vf \partial x\ +\ \left[ v \frac{\partial v}{\partial x} \right]_\Omega
 \end{equation}
 introduzindo novas notações para ambos os lados da função temos: 
\begin{align}
 a(v,u)\ = & \int^l_{0} \left(\frac{\partial v}{\partial x}\frac{\partial u}{\partial x} + \lambda v u\right) \partial x\\
 f(v)\ = & \int^{l}_0 v f \partial x + \left[v \frac{\partial v}{\partial x}\right]^{l}_{0} 
\end{align}
 A equação \ref{eq:nova_eqdif} pode ser reescrita como:
 \begin{equation}
 a(u,v) = f(v)
 \end{equation}
 No estudo de estruturas,  $a(u,u)$ é chamado de \emph{energia de deformação} e o espaço de funções na qual temos uma \emph{deformação} finita em $\Omega$, chamamos de \emph{espaço de energia} , denotado por $E(\Omega)$:
\begin{equation}
 E(\Omega) = \lbrace u | a(u,v) < \infty \rbrace
\end{equation}
  Associado a esse espaço, temos uma norma associada, denominada por $||u||_E$:
\begin{equation}
 ||u||_{E} = \sqrt{a(u,v)}
\end{equation}
 Funções que pertencem a esse espaço são chamados de funções $H^1$ e satisfazem a condição:
 \begin{equation}
 H^1(\Omega) = \bigg\{ u| \int_\Omega (|\nabla u|^2 + |u|^2) \partial \Omega  \bigg\}
 \end{equation}
  Assim temos o espaço da solução dado por:
\begin{equation}
U = \lbrace u| u \in H^1\ , u(0) = g_D  \rbrace
\end{equation}
 E o espaço da função teste é dado por :
\begin{equation}
V = \lbrace v | v \in H^1, v(0)=0 \rbrace
\end{equation}

 Assim, a formulação fraca do problema fica em encontrar $u \in U$ dado:
 \begin{equation}
 a(v,u)\ =\ f(v)\ ,\ \forall\ v \in V 
 \end{equation}
 Assim, temos o problema de soluções infinitas, pois ambos os espaços, U e V, contêm um número infinito de funções. Assim, selecionamos os subespaços $U^\delta$ ($ U^\delta \subset U$) e $V^\delta$ ($ V^\delta \subset V$) que contem um número finito de funções.
 