\chapter{Conclusões}

 Com esse trabalho de formatura, consegui  passar por algumas das diversas disciplinas que me foram apresentadas da graduação de Matemática Numérica, sendo essas a parte numérica onde tive que rever métodos como a interpolação numérica de funções, entender sobre o erro da aproximação envolvido na escolha dos pontos de interpolação com o fenômeno de Runge. Também aprendi conceitos novos como o caso do método dos resíduos ponderados que apesar de ser uma generalização do método dos mínimos quadrados, me fez ter uma melhor compreensão dessa ferramenta poderosa de minimização do erro. 
 \subsection{Aprendizado na teoria}
 Métodos novos que não foram contemplados na minha graduação como o método espectral e o método dos elementos finitos foram interessantes de se entender apesar da minha dificuldade de abstração nos momentos de representação matricial e no remapeamento de funções globais em locais. Além de um entendimento sobre os polinômios conhecidos durante o curso aprendi que são ramificações do polinômio de Jacobi. Juntando esses conhecimentos obtive uma nova técnica para resolução de equações diferenciais de primeiro grau que me forneceu uma boa base para alguma futura resolução de equações em espaços de maior dimensão.

 \subsection{Implementação do código} 
 Com a linguagem júlia, aprendi novos paradigmas de programação como programação orientada a objetos, assim como adquiri um maior respeito com os desenvolvedores das bibliotecas \href{http://www.netlib.org/lapack/}{LAPACK} e \href{http://www.netlib.org/blas/}{BLAS}, ambas bibliotecas de subrotinas para a solução de problemas de algebra linear, pois nelas vi que em comparação a algumas rotinas implementadas por mim, eram muito mais rápidas em execução e resolviam algumas das questões em erros de pontos flutuantes. 
 
 Assim como aprendi como funciona a implementação de uma biblioteca que poderá ser utilizada para um público maior no futuro, tendo assim um maior cuidado com a documentação do código. Utilizei também  git e o \href{http://www.github.com/}{Github}, respectivamente ferramenta de controle de versão e servidor online que foram utilizadas para atualizar as mudanças no código e usadas por mim para reportar algum erro ou bug do código para o Dr. Paulo Jabardo, inclusive utilizei essas ferramentas para a criação desse trabalho de conclusão evitando possíveis perdas de arquivo por queima de HD ou deletar acidentalmente o arquivo final de minha monografia.

 Juntando ambas as teoria matemática por trás do método e a computação, pude perceber as dificuldades da implementação de um método numa nova linguagem no qual a graduação me limitou em apresentar a teoria como foi o caso da Algebra Linear e aplicações de Algebra Linear, na qual descobri que há diversos problemas como de ponto flutuante e tempo de execução.
 
 Esse trabalho de conclusão me permitiu experimentar melhor o dia a dia de um profissional da área com algumas das dificuldades e felicidades que é a aprendizagem e implementação em código de um novo método matemático.
 