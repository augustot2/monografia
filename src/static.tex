\chapter{Static Condensation [Capítulo avulso]}
\section{Condensação estática}
  Quando usanmos o \emph{Método de elementos espectrails} para resolver equações diferenciais, é comum ter funções de base para os modos  \textbf{internos} (i) e os modos de \textbf{fronteira} (b) de cada elemento. Utilizando a técnica de \emph{condensação estática} podemos usar essa estrutura para resolver sistemas lineares do tipo:
  \begin{equation}
   A x = f
  \end{equation}
onde rearranjamos os vetores x e f de maneira a ficar primeiro os coeficientes de fronteira e depois os coeficientes internos:
  \begin{equation}
    x = \left\{\begin{matrix} x_b \\ x_i \\ \end{matrix}\right\}\ ,\ f =  \left\{\begin{matrix} f_b \\ f_i \\ \end{matrix}\right\}\
  \end{equation}
E com esse rearranjo, temos a matriz A como:
\begin{equation}
A = \left[ 
\begin{matrix} 
A_{bb} & A_{bi} \\
A_{ib} & A_{ii} \\
\end{matrix}\right]
\end{equation}
onde cada $A_{\cdot \cdot}$ é uma sub matriz de A, assim, temos o sistema linear como:
\begin{equation}
A\cdot x  = \left[ 
\begin{matrix} 
A_{bb} & A_{bi} \\
A_{ib} & A_{ii} \\
\end{matrix}\right] \cdot \left\{
\begin{matrix} x_b \\ x_i \\ 
\end{matrix}\right\} = 
\left\{\begin{matrix} f_b \\ f_i \\\end{matrix}\right\} 
\end{equation}
Para resolver esse sistemas fazemos:
\begin{equation}
x_i = A_{ii}^{-1} f_i - A_{ii}^{-1} A_{ib} x_b
\end{equation}
Que substituindo $x_i$ na equação de  $x_b$, obtemos:
\begin{equation}
\left( A_{bb} - A_{bi}A_{ii}^{-1} A_{ib} \right) x_b = f_b - A_{bi}A_{ii}^{-1} f_i\\
A'_{bb} \cdot x_b = f'_b
\end{equation}
onde:
\begin{align}
& A'_{bb} = A_{bb} - A_{bi}A_{ii}^{-1} A_{ib}\\ 
& f'_b = f_b - A_{bi}A_{ii}^{-1} f_i
\end{align}
Assim, encontramos $x_b$ e $x_i$ da equação.



\section{Método dos resíduos ponderados}
 O método dos resíduos ponderados nos permite encontrar soluções para problemas nos qual queremos minimizar o erro de aproximação de um problema. No caso, estamos interessados na equação diferencial.
 \begin{equation}
    L(u) = \frac{\partial^2 u}{\partial x^2} + \lambda u + f = 0 \\
    L(u) = 0
 \end{equation}
 Aproximando $u$ por $u^\delta$ obtemos um erro associado ao operador L:
 \begin{equation}
 	u^\delta(x) = u(x_0) + \sum^{N_{dof}}_{i = 1} u(x_i) \Phi(x)\\
    L(u^\delta) = R(u^\delta)
\end{equation}
Nosso problema está na minimização desse erro $R(u^\delta)$. Para isso, escolhemos uma função teste $v_j$ onde $j = 1,2,\dots, N_{dof}$, tal que seu produto interno com o erro seja o menor possível.
\begin{equation}
    <v_j,R(u^\delta)> = \int_{\Omega} v_j(x)\ R(x) \partial x,\ \forall j = 1,2,3,\dots,N_{dof}\\
\end{equation}
Assim, obtemos um sistema linear no qual escolhido a melhor função teste $v_j$, temos a melhor aproximação $u^\delta$.

\begin{align}
    <v_j,R(u^\delta)> = \int_{\Omega} v_j(x)\ R(u^\delta(x)) \partial x,\ \forall j = 1,2,3,\dots,N_{dof}\\
    <v_1,R(u^\delta)> = \int_{\Omega} v_1(x)\ R(u^\delta(x)) \partial x\\
    <v_2,R(u^\delta)> = \int_{\Omega} v_2(x)\ R(u^\delta(x)) \partial x\\
    <v_3,R(u^\delta)> = \int_{\Omega} v_3(x)\ R(u^\delta(x)) \partial x\\
       \vdots \   \ \vdots \ \vdots \ \vdots  \\
    <v_{N_{dof}},R> = \int_{\Omega} v_{N_{dof}}(x)\ R(x) \partial x
\end{align}