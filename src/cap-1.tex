\chapter{Capítulo I}
\label{cap:I}
\section{Método espectral}

 Método espectral é um método poderoso usado para solução de equações diferencial parcial. Diferentemente do método das diferenças finitas, que considera apenas os pontos próximos do ponto que queremos computar chamada de método \emph{local}, o método espectral considera todo o domínio, sendo assim um método \emph{global}. Essa técnica tem mais precisão pois converge exponencialmente diferente do método local. É preferível a utilização desse método quando a solução varia em função do \textit{tempo} e \textit{espaço}.

\section{Interpolação}
 A interpolação de uma função $f(x)$ por um polinômio trigonométrico ou não, de grau $n$, $P_{n}(x)$ e que satisfaça:

\begin{equation}
	P_n (x_i) = f(x_i) \ i = 1,2,...,\emph{n+1}
\end{equation}

 Onde $f(x_i)$ é a função $f$ pré-calculada nos pontos $x_i$. A escolha desses pontos $x_i$ ainda será explicada.
 
\subsection{Interpolação polinomial}

 Antes do uso de calculadoras e computadores, um método de estimar o valor de uma $f$ num ponto, eram utilizados tabelas com valores de pré-calculados. A maneira mais simples de entender é a estimação do valor da função em um ponto intermediário entre dois pontos conhecidos é o uso da interpolação \emph{Linear}.

\begin{equation}
	f(x) \approx \frac{x - x_1}{x_0 - x_1}f(x_0)  + 							 \frac{x - x_0}{x_1 - x_0}f(x_1)
\end{equation}
 
\includegraphics[scale=0.4]{figuras/interpolacao_linear.png}
 
 Para fazermos essa interpolação para $n$ pontos conhecidos aproximamos uma função usando o polinômio base de \emph{Lagrange}.
\begin{equation}
C_i(x) = \prod_{j = 0 \\ j \neq i}^{N} \frac{x - x_j}{x_i - x_j} 
\end{equation}
 A interpolação de \emph{Lagrange} é dada por :
\begin{equation}
 P_n(x) \equiv \sum_{i = 0}^{N} f(x_i)C_i(x) 
\end{equation}
 Obedecendo que $P_n(x_i) = f(x_i)$. Apesar dos pontos interpoladores equidistante serem comumente utilizados, não há restrições, podendo até mesmo estar fora de ordem.
 
\subsection{Fenômeno de Runge}
 





