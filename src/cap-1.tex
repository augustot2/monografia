\chapter{Capítulo I}
\label{cap:I}
\section{Método espectral}

 A ideia geral do método espectral é a aproximação de uma função qualquer, $f(x) : \mathbb{R} \rightarrow \mathbb{R}$, usando um polinômio de alta ordem $\mathit{n}$, $P_{n}(x)$. Sabendo os pontos $(x_k,y_k)$, definimos então o polinômio:
\begin{equation}
	P(x_k) = y_k, k = 1,...,N
\end{equation} 

 Para uma equação com 2 pontos $(x_1,x_2,y_1,y_2)$ podemos definir um Polinômio de grau 1 que obedece $P(x_k) = y_k, k = 1,2$. Assim, para $N+1$ 
pontos, podemos aproximar por um polinômio de grau menor ou igual que $N$.

 Podemos usar diversos polinômios, um dos mais conhecidos é o polinômio de \textit{Lagrange} :
 
 \begin{equation}
 P(x) = \sum_{k = 1}_{k ^{N} \frac{x - x_j}{x_k - x_j}y_k
 \end{equation}
 
 
Exemplo 


Exemplo 
 
