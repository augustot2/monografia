%% ------------------------------------------------------------------------- %%
\chapter{Introdução}
\label{cap:introducao}
\section{História}
O método espectral surgiu como uma ferramenta de alto poder computacional em mecânica de fluídos, proposto em 1944 por Blinova \cite{blinova1944}, implementado em 1954 por Silberman \cite{silberman}, praticamente abandonado no meio da década de 60 e ressurgindo em 1969-1970 por Orszag \cite{Orszag70}, Eliason e Manchenhauer  e Rasmussen \cite{eliasenauermussen}, foi desenvolvido para aplicações especializadas. No entanto,  somente em 1977 foi formalizado matematicamente por Gottlieb e Orszag \cite{Orslieb77}.

Originalmente o método espectral foi promovido por meteorologistas no estudo de modelos globais de tempo e especialistas em dinâmica de fluídos estudando turbulências isotrópicas. Desde a década de 80 até hoje o  estudo na área de CFD (Computational Fluid Dynamics- dinâmica dos fluidos computacional) tem crescido lado a lado ao avanço computacional que o tornou possível.
\section{Julia}
Desenvolvida em código aberto a partir de 2009 e lançada em fevereiro de 2012, a linguagem de programação Julia vem ganhando notoriedade entre as diversas áreas científicas como estatística, biologia e na computação numérica.

A linguagem Julia, foi desenvolvida afim de juntar os aspectos positivos de linguagens de alto nível como python que tem polimorfismo, tipagem dinâmica, fácil entendimento do código e junto a isso o alto desempenho em tarefas de maiores complexidades que linguagens de baixo nível como C tem, porém sem o  alto custo de produtividade do usuário para sua implementação.

Assim, utilizando as bibliotecas desenvolvidas pelo  Dr. Paulo José Saiz Jabardo \href{www.github.com/pjabardo/HPFEM.jl}{HPFEM.jl}, testada por mim e \href{www.github.com/pjabardo/Jacobi.jl}{Jacobi.jl} , resolverei alguns dos problemas de interpolação e na resolução de equações diferenciais em 1D apresentados nos próximos capítulos.

\pagebreak
%% ------------------------------------------------------------------------- %%
%\section{Contribuições}
%\label{sec:contribucoes}
%
%As principais contribuições deste trabalho são as seguintes:
%
%\begin{itemize}
%  \item
%

%\end{itemize}

%\section{Organização do Trabalho}
%\label{sec:organizacao_trabalho}

% No Capítulo~\ref{cap:conceitos}, apresentamos os conceitos ... Finalmente, no
% Capítulo~\ref{cap:conclusoes} discutimos algumas conclusões obtidas neste
% trabalho. Analisamos as vantagens e desvantagens do método proposto ...

% As sequências testadas no trabalho estão disponíveis no Apêndice \ref{ape:sequencias}.
