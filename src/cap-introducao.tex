%% ------------------------------------------------------------------------- %%
\chapter{Introdução}
\label{cap:introducao}
\section{História}
O método espectral surgiu como uma ferramenta de alto poder computacional em mecânica de fluídos, proposto em 1944 por Blinova \cite{blinova1944}, implementado em 1954 por Silberman \cite{silberman}, praticamente abandonado no meio da década de 60 e ressurgindo em 1969-1970 por Orszag \cite{Orszag70}, Eliason e Manchenhauer  e Rasmussen \cite{eliasenauermussen}, foi desenvolvido para aplicações especializadas. No entanto,  somente em 1977 foi formalizado matematicamente por Gottlieb e Orszag \cite{Orslieb77}.

Originalmente o método espectral foi promovido por meteorologistas no estudo de modelos globais de tempo e especialistas em dinâmica de fluídos estudando turbulência isotrópica. Desde a década de 80 até hoje o  estudo na área de CFD (Computational Fluid Dynamics- dinâmica dos fluidos computacional) tem crescido lado a lado ao avanço computacional que o tornou possível.
\section{A monografia}
 Neste trabalho, irei realizar algumas das técnicas do método espectral para a resolução de equações diferenciais em 1D (uma dimensão) apresentados por G. Karniadakis e S. Sherwin \citep{book:karniadakis}. Assim abordarei tópicos de métodos numéricos como  interpolação, diferenciação e integração. Apresentarei o do fenômeno \emph{Runge}, no qual vemos a melhor escolha de pontos para a resolução das técnicas anteriores.
 
 A seguir apresento o  método dos resíduos ponderados e uma de suas ramificações chamadas de  método de Galerkin que são técnicas de minimização do erro, essa última sendo a técnica que será utilizada para a resolução de equações diferenciais. Com o método apresentado para um polinômio de grau elevado, será introduzida a técnica de elementos finitos e a forma matricial do problema e a utilização de um solver desse sistema linear chamado de condensação estática.
 
 Por fim, no último capítulo, serão apresentadas simulações dos métodos apresentados anteriormente, como a interpolação de uma função, visualizando o fenômeno \emph{Runge} e a solução de equações diferenciais para casos de conhecidas os tipos de fronteiras (\emph{Dirichlet} ou \emph{Neumann}). Também, verificaremos o tipo de convergência do erro máximo da solução alternando o grau do polinômio interpolador e o número de elementos. Onde todos os gráficos e métodos nesse trabalho foram implementados na linguagem computacional chamada de Julia.
  
 
\section{A linguagem Julia}
\begin{figure}[H]
\centering
\includegraphics[width=0.4\textwidth,left]{figuras/julia.png}
\end{figure}
Desenvolvida em código aberto a partir de 2009 e lançada em fevereiro de 2012, a linguagem de programação Julia vem ganhando notoriedade entre as diversas áreas científicas como estatística, biologia e na computação numérica.

A linguagem \href{http://julialang.org/}{Julia}, foi desenvolvida afim de juntar os aspectos positivos de linguagens de alto nível como python que tem polimorfismo, tipagem dinâmica, fácil entendimento do código e junto a isso o alto desempenho em tarefas de maiores complexidades que linguagens de baixo nível como C tem, porém sem o  alto custo de produtividade do usuário para sua implementação. Procurando fontes, percebi que o motivo desse nome aparentemente não existe: ''- apenas parecia um bom nome para uma nova linguagem computacional''\  Jeff Bezanson, um dos criadores da linguagem.

Assim, utilizando as bibliotecas desenvolvidas por mim e pelo  Dr. Paulo José Saiz Jabardo \href{www.github.com/pjabardo/HPFEM.jl}{HPFEM.jl} e \href{www.github.com/pjabardo/Jacobi.jl}{Jacobi.jl} , resolverei alguns dos problemas de interpolação e na resolução de equações diferenciais em 1D apresentados nos próximos capítulos.
 
\pagebreak

%% ------------------------------------------------------------------------- %%
%\section{Contribuições}
%\label{sec:contribucoes}
%
%As principais contribuições deste trabalho são as seguintes:
%
%\begin{itemize}
%  \item
%

%\end{itemize}

%\section{Organização do Trabalho}
%\label{sec:organizacao_trabalho}

% No Capítulo~\ref{cap:conceitos}, apresentamos os conceitos ... Finalmente, no
% Capítulo~\ref{cap:conclusoes} discutimos algumas conclusões obtidas neste
% trabalho. Analisamos as vantagens e desvantagens do método proposto ...

% As sequências testadas no trabalho estão disponíveis no Apêndice \ref{ape:sequencias}.
