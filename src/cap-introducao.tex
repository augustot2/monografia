%% ------------------------------------------------------------------------- %%
\chapter{Introdução}
\label{cap:introducao}

O método espectral surgiu como uma ferramenta de alto poder computacional em mecânica de fluídos, proposto em 1994 por Blinovo, implementado em 1954 por Sylberman, praticamente abandonado no meio da década de 60 e ressurgindo em 1969-1970 por Orzszag e Eliason, Manchenhauer e Rasmussen, foi desenvolvido para aplicações especializadas. No entanto, somentem em 1977 foi formalizado matematicamente por Gottlieb e Orszag em 1980.
Originalmente o método espectral foi promovido por meteorologistas no estudo de modelos globais de tem e especialistas em dinâmica de fluídos
%% ------------------------------------------------------------------------- %%
%\section{Contribuições}
%\label{sec:contribucoes}
%
%As principais contribuições deste trabalho são as seguintes:
%
%\begin{itemize}
%  \item
%

%\end{itemize}

%\section{Organização do Trabalho}
%\label{sec:organizacao_trabalho}

% No Capítulo~\ref{cap:conceitos}, apresentamos os conceitos ... Finalmente, no
% Capítulo~\ref{cap:conclusoes} discutimos algumas conclusões obtidas neste
% trabalho. Analisamos as vantagens e desvantagens do método proposto ...

% As sequências testadas no trabalho estão disponíveis no Apêndice \ref{ape:sequencias}.
