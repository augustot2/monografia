%% ------------------------------------------------------------------------- %%
\chapter{Introdução}
\label{cap:introducao}
\section{História}
O método espectral surgiu como uma ferramenta de alto poder computacional em mecânica de fluídos, proposto em \corr{1994 por Blinova O ANO ESTÁ CERTO???}, implementado em 1954 por Sylberman, praticamente abandonado no meio da década de 60 e ressurgindo em 1969-1970 por Orzszag e Eliason, Manchenhauer e Rasmussen, foi desenvolvido para aplicações especializadas. No entanto, \corr{somente em 1977 foi formalizado matematicamente por Gottlieb e Orszag em 1980 DECIDA-SE QUAL O ANO}.  \corr{Sugestão PJ: usa o bibtex e coloca as referências. É o correto e não há ambiguidade}

Originalmente o método espectral foi promovido por meteorologistas no estudo de modelos globais de tempo e especialistas em dinâmica de fluídos estudando turbulências isotrópicas. Desde a década de 80 até hoje o  estudo na área de CFD (Computational Fluid Dynamics- dinâmica dos fluidos computacional) tem crescido lado a lado ao avanço computacional que o tornou possível.
\section{Julia}
 Para a implementação do método usaremos como ferramenta de estudo a linguagem de alto nível, \emph{Julia}, que por ser dinâmica e de excelente desempenho  computacional será essencial para execução dos cálculos. \corr{Será essencial??? Até hoje não se fazia nada? Porque você acha julia interessante?. Leia o que está na pag. da julia}.

 Apesar de nova, a linguagem criada no MIT vem sendo rapidamente acolhida pela comunidade científica e assim, com seu código open-source, ele é diariamente atualizado e possui um número de  bibliotecas em crescente ascensão.


%% ------------------------------------------------------------------------- %%
%\section{Contribuições}
%\label{sec:contribucoes}
%
%As principais contribuições deste trabalho são as seguintes:
%
%\begin{itemize}
%  \item
%

%\end{itemize}

%\section{Organização do Trabalho}
%\label{sec:organizacao_trabalho}

% No Capítulo~\ref{cap:conceitos}, apresentamos os conceitos ... Finalmente, no
% Capítulo~\ref{cap:conclusoes} discutimos algumas conclusões obtidas neste
% trabalho. Analisamos as vantagens e desvantagens do método proposto ...

% As sequências testadas no trabalho estão disponíveis no Apêndice \ref{ape:sequencias}.
