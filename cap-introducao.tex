%% ------------------------------------------------------------------------- %%
\chapter{Introdu��o}
\label{cap:introducao}

Escrever bem � uma arte que exige muita t�cnica e dedica��o. H� v�rios bons livros
sobre como escrever uma boa disserta��o ou tese. Um dos trabalhos pioneiros e mais
%conhecidos nesse sentido � o livro de \citet{eco:09} intitulado 
\emph{Como se faz uma tese}; � uma leitura bem interessante mas, como foi escrito 
em 1977 e � voltado para teses de gradua��o na It�lia, n�o se aplica tanto a n�s.

Para a escrita de textos em Ci�ncia da Computa��o, o livro de Justin Zobel, 
\emph{Writing for Computer Science} \citep{zobel:04} � uma leitura obrigat�ria. 
O livro \emph{Metodologia de Pesquisa para Ci�ncia da Computa��o} de 
\citet{waz:09} tamb�m merece uma boa lida.
J� para a �rea de Matem�tica, dois livros recomendados s�o o de Nicholas Higham,
\emph{Handbook of Writing for Mathematical Sciences} \citep{Higham:98} e o do criador
do \TeX, Donald Knuth, juntamente com Tracy Larrabee e Paul Roberts, 
\emph{Mathematical Writing} \citep{Knuth:96}.

O uso desnecess�rio de termos em lingua estrangeira deve ser evitado. No entanto,
quando isso for necess�rio, os termos devem aparecer \emph{em it�lico}.
